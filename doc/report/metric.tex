\chapter{Planar Universal Geometry}\label{chap:metric}

We begin by developing some results about points and lines in the plane when taken with an arbitrary quadratic form as the metric.
Some of the results presented below are well known results from Euclidean geometry while others are completely novel.
Those non-original results are included for completeness, so that the reader may have a comprehensive set of results presented in a consistent notation.
Many of the definitions make reference to a general field $\mathbb{F}$.
We will take as given that this field is not of characteristic two, as otherwise many of the definitions would break down.

\section{Notation and Definitions}

\begin{definition} Given a field $\mathbb{F}$, a \emph{point} is any element $X \in \mathbb{F}\times\mathbb{F}$ and we write $X = [x, y]$ where $x, y \in \mathbb{F}$.
\end{definition}
An alternative representation of a point is as a two element column vector.
The vector representation allows us to use standard linear algebra notation when doing calculations.
\begin{definition} The vector representation of a point $X$ is
\begin{eqnarray}
\vec{X} & = & \begin{pmatrix} x \\ y \end{pmatrix}.
\end{eqnarray}
\end{definition}

\begin{definition} Given a field $\mathbb{F}$, a \emph{line} is the locus of points $[x, y] \in \mathbb{F}\times\mathbb{F}$ which satisfy the equation
\begin{eqnarray}
ax + by + c & = & 0,
\end{eqnarray}
where $a, b, c \in \mathbb{F}$ and $a$, $b$ are not both zero.
The standard representation of this line is $\langle a\!:\!b\!:\!c \rangle$, which emphasises the fact that what is important is the ratio between the values of $a$, $b$ and $c$.
The lines $\langle a\!:\!b\!:\!c \rangle$ and $\langle \lambda a\!:\! \lambda b\!:\! \lambda c \rangle$, where $\lambda \in \mathbb{F}$ is non-zero, are equivalent, as they represent the same locus of points.
\end{definition}

We can also represent a line as a two element column vector.
This representation does not completely specify the line and so more care must be taken with its use.

\begin{definition} The vector representation of a line $l$ is
\begin{eqnarray}
\vec{l} & = & \begin{pmatrix} -b \\ a \end{pmatrix}.
\end{eqnarray}
\end{definition}

We also note here that the forms $\langle a\!:\!b\!:\!c \rangle$ and $\langle \lambda a\!:\! \lambda b\!:\! \lambda c \rangle$ represent the same line, yet have different vector representations.
As such when using the vector representation of a line we must be careful not to use the scaling properties of the standard representation, as this will lead to inconsistencies.

To perform any kind of geometrical measurement, we require a way to define the quantities being measured.
By changing the way quantities are measured, we can change the properties of the geometry itself.
In planar universal geometry measurement is done in terms of a quadratic form, known as the metric.

\begin{definition} Given a field $\mathbb{F}$, a \emph{metric} $G$ is a symmetric, non-singular two-by-two matrix over the field, i.e.
\begin{eqnarray}
G & = & \begin{pmatrix} a & b \\ b & c \end{pmatrix}\\
\Delta_G & = & ac - b^2\nonumber\\
 & \neq & 0
\end{eqnarray}
where $a, b, c \in \mathbb{F}$.
\end{definition}

To make reading the following proofs simpler, a standard naming convention has been adopted for points, lines and metrics.
A points will be denoted as $X_n$, and it is assumed that its components are $[x_n, y_n]$.
A line will be denoted as $l_n$, and it is assumed that its general form is $\langle a_n\!:\!b_n\!:\!c_n \rangle$.
The metric $G$ will always be assumed to be $\begin{pmatrix} a & b \\ b & c\end{pmatrix}$.
Unless otherwise specified it can be assumed that this convention is being followed.

We will often associate a geometry with its metric, and thus the phrases ``a geometry with a metric $G$'' and ``a geometry $G$'' will be used interchangeably.
Planar universal geometry allows many of the ideas of Euclidean geometry to be generalised.
The dot product is one such operation.

\begin{definition} The \emph{metric dot product} in a geometry with metric $G$ is defined as
\begin{eqnarray}
\vec{v}\cdot_G\vec{u} & = & \vec{v}^TG\vec{u}
\end{eqnarray}
where $\vec{v}$ and $\vec{u}$ are vectors which may represent either lines or points.
\end{definition}

The definition of a metric dot product leads naturally to the idea of a metric norm.

\begin{definition} The \emph{metric norm} of a vector $\vec{q}$ in a geometry with metric $G$ is defined as
\begin{eqnarray}
\|\vec{q}\|_G & = & \vec{q}^TG\vec{q}.
\end{eqnarray}
\end{definition}
It will often be clear from the context that the metric dot product or metric norm is being used, in which case the $_G$ subscript will be dropped to ease readability.


\begin{definition} A line $l$ is a \emph{null line} in the geometry $G$ if the norm of the vector representation of the line is zero, i.e. $\|\vec{l}\|_G = 0$.
\end{definition}

\begin{definition} A point $X$ is a \emph{null point} in the geometry $G$ if the norm of the vector representation of the point is zero, i.e. $\|\vec{X}\|_G = 0$.
\end{definition}

In standard Euclidean geometry, the separation of any two points is given in terms of distance, which is calculated using a metric function such as $d(X_0, X_1) = \sqrt{\|\vec{X_1} - \vec{X_0}\|}$.
Since we wish to use only field operations, we replace the concept of distance with that of quadrance.

\begin{definition} The \emph{quadrance} between two points $X_0$ and $X_1$ in a geometry with metric $G$ is defined as
\begin{eqnarray}
Q(X_0, X_1)_G & = & \left\|\vec{X_0} - \vec{X_1}\right\|_G.
\end{eqnarray}
\end{definition}

As is the case with distance, the standard measure of separation of lines, the angle between them, cannot be defined in terms of field operations.
We replace the notion of angle with that of spread.

\begin{definition}The \emph{spread} between two non-null lines $l_1$ and $l_2$ in a geometry with metric $G$ is defined as
\begin{eqnarray}
s & = & 1 - \frac{\left(\vec{l_1}\cdot_G \vec{l_2}\right)^2}{\|\vec{l_1}\|_G\|\vec{l_2}\|_G}.
\end{eqnarray}
\end{definition}
If either $l_1$ or $l_2$ are null lines then the spread between these lines is not defined.
The definition above is essentially identical to that for spread in universal geometry.
We can use the fact that we are working in two dimensions to simplify this further.

\begin{theorem}\label{th:alt-spread}The spread between the non-null lines $l_1 = \langle a_1\!:\!b_1\!:\!c_1 \rangle$ and $l_2 = \langle a_2\!:\!b_2\!:\!c_2 \rangle$ in a geometry $G$ is
\begin{eqnarray}
s & = & \Delta_G\frac{(a_1b_2 - a_2b_1)^2}{\|\vec{l_1}\|\|\vec{l_2}\|}.
\end{eqnarray}
\end{theorem}
\begin{proof}We first evaluate the denominator in the spread formula to get
\begin{eqnarray*}
\|\vec{l_1}\|\|\vec{l_2}\| & = & (ab_1^2 -2ba_1b_1 + ca_1^2)(ab_2^2 - 2ba_2b_2 + ca_2^2)\\
 & = & a^2b_1^2b_2^2 +4b^2a_1a_2b_1b_2 + c^2a_1^2a_2^2 -2abb_1b_2(b_1a_2 + a_1b_2) + \\
 &  & ac(a_2^2b_1^2 + a_1^2b_2^2) -2bca_1a_2(a_2b_1 + a_1b_2).
\end{eqnarray*}
Now, the numerator is
\begin{eqnarray*}
\left(\vec{l_1}\cdot \vec{l_2}\right)^2 & = & \left(b_1(ab_2 - ba_2) - a_1(bb_2 - ca_2)\right)^2\\
 & = & \left(ab_1b_2 - b(a_1b_2 + a_2b_1) + ca_1a_2\right)^2\\
 & = & a^2b_1^2b_2^2 + b^2(a_1b_2 + a_2b_1)^2 + c^2a_1^2a_2^2 -2abb_1b_2(a_1b_2 + a_2b_1) + \\
 &   & 2aca_1a_2b_1b_2 - 2bca_1a_2(a_1b_2 + a_2b_1).
\end{eqnarray*}
Using the result found for the denominator we can write this as
\begin{eqnarray*}
\left(\vec{l_1}\cdot \vec{l_2}\right)^2 & = & \|\vec{l_1}\|\|\vec{l_2}\| + b^2(a_1b_2 + a_2b_1)^2 + 2aca_1a_2b_1b_2 - 4b^2a_1a_2b_1b_2 - \\
 &  & ac(a_1^2b_2^2 + a_2^2b_1^2)\\
 & = & \|\vec{l_1}\|\|\vec{l_2}\| + (b^2 - ac)(a_1b_2 - a_2b_1)^2\\
 & = & \|\vec{l_1}\|\|\vec{l_2}\| - \Delta_G(a_1b_2 - a_2b_1)^2.
\end{eqnarray*}
The spread formula now becomes
\begin{eqnarray*}
s & = & 1 - \frac{\|\vec{l_1}\|\|\vec{l_2}\| - \Delta_G(a_1b_2 - a_2b_1)^2}{\|\vec{l_1}\|\|\vec{l_2}\|}\\
 & = & \Delta_G\frac{(a_1b_2 - a_2b_1)^2}{\|\vec{l_1}\|\|\vec{l_2}\|}.
\end{eqnarray*}
\end{proof}

\begin{definition}[Archimedes function]
The function $A : \mathbb{F}\times\mathbb{F}\times\mathbb{F} \to \mathbb{F}$ defined as
\begin{eqnarray}
a, b, c & \mapsto & (a + b + c)^2 - 2(a^2 + b^2 + c^2)\label{eq:arch}
\end{eqnarray}
is called \emph{Archimedes function}\cite{thebook}.
\end{definition}

\section{Properties of Lines}

Many of the properties of lines in rational trigonometry extend naturally to planar universal geometry.
In the following definitions we do not use the metric of the geometry explicitly and so equivalent results can be found in the study of rational trigonometry\cite{thebook}.

\begin{definition} Two lines are \emph{perpendicular} when the spread between them is equal to one.
\end{definition}

\begin{corollary} If two lines are perpendicular then their metric dot product is zero.
\end{corollary}
\begin{proof}This follows directly from the definition of the spread between two lines.
\end{proof}

\begin{definition} Two lines $l_1$ and $l_2$ are \emph{parallel} if $a_1b_2 - a_2b_1 = 0$.
\end{definition}
\begin{corollary} Two non-null lines $l_1$ and $l_2$ are parallel if and only if the spread between them is zero.
\end{corollary}
\begin{proof}This follows directly from Theorem \ref{th:alt-spread}.
\end{proof}

\begin{definition} A \emph{point of intersection} of two lines is a point which lies on both lines.
\end{definition}
\begin{theorem}\label{th:intersect}If two lines $l_1$ and $l_2$ are not parallel then they intersect at the point $X = \left[\frac{b_1c_2 - b_2c_1}{a_1b_2 - a_2b_1}, \frac{a_2c_1 - a_1c_2}{a_1b_2 - a_2b_1}\right]$.
\end{theorem}
\begin{proof}We need to solve the following two equations for $x$ and $y$:
\begin{eqnarray*}
a_1x + b_1y + c_1 & = & 0\\
a_2x + b_2y + c_2 & = & 0.
\end{eqnarray*}
Writing these in matrix form we get
\begin{eqnarray*}
\begin{pmatrix}a_1 & b_1 \\ a_2 & b_2 \end{pmatrix}\begin{pmatrix}x \\ y \end{pmatrix} & = & -\begin{pmatrix}c_1 \\ c_2 \end{pmatrix}\\
\begin{pmatrix} x \\ y \end{pmatrix} & = & -\begin{pmatrix} a_1 & b_1 \\ a_2 & b_2 \end{pmatrix}^{-1}\begin{pmatrix}c_1 \\ c_2 \end{pmatrix}\\
 & = & \frac{-1}{a_1b_2 - a_2b_1}\begin{pmatrix} b_2 & -b_1 \\ -a_2 & a_1 \end{pmatrix}\begin{pmatrix}c_1 \\ c_2 \end{pmatrix}
\end{eqnarray*}
\end{proof}

\begin{definition} Given a point $X_0$ and a line $l_1$, the \emph{residue} of the point with respect to the line is
\begin{eqnarray}
l_1(X_0) & = & a_1x_0 + b_1y_0 + c_1.
\end{eqnarray}
\end{definition}
\begin{corollary}If the point $X$ lies on the line $l$ then $l(X) = 0$.
\end{corollary}
\begin{proof}
This is an immediate consequence of the definition of a line as the locus of points satisfying the equation $ax + by + c = 0$.
\end{proof}

\section{Properties of Points}

Given two distinct points, the line which passes through them is defined independently of the geometry.
As such the first theorem regarding points is a well known result of Euclidean geometry.
\begin{theorem} The line which passes through the points $X_0$ and $X_1$ is $\langle a_1\!:\!b_1\!:\!c_1\rangle$ where
\begin{eqnarray}
a_1 & = & y_1 - y_0\\
b_1 & = & x_0 - x_1\\
c_1 & = & x_1y_0 - x_0y_1.
\end{eqnarray}
\end{theorem}
\begin{proof}If we assume the line has the form given above and then take the residues of $X_0$ and $X_1$ we get
\begin{eqnarray*}
l_1(X_0) & = & (y_1 - y_0)x_0 + (x_0 - x_1)y_0 + (x_1y_0 - x_0y_1)\\
 & = & 0\\
l_1(X_1) & = & (y_1 - y_0)x_1 + (x_0 - x_1)y_1 + (x_1y_0 - x_0y_1)\\
 & = & 0.
\end{eqnarray*}
Since the residue of both points is zero they both lie on the line, as required.
\end{proof}

The quadrance between two points is defined in terms of the metric of the geometry.
As such we expect that the set of points which are equiquadrance from two given points to depend on the metric.
As is the case with Euclidean geometry, this set of points turns out to always be a line.

\begin{definition} The \emph{equiquadrance line} in a geometry $G$ between two points $X_0$ and $X_1$ is the locus of points such that $Q(X, X_0) = Q(X, X_1)$.
\end{definition}

\begin{theorem}
Given two points $X_0$, $X_1$ in a geometry $G$, the equation of the equiquadrance line is $\langle a_1\!:\!b_1\!:\!c_1\rangle$ where
\begin{eqnarray}
\begin{pmatrix} a_1 \\ b_1 \end{pmatrix} & = & 2G(\vec{X_0} - \vec{X_1})\\
c_1 & = & \vec{X_1}^2 - \vec{X_0}^2.
\end{eqnarray}
\end{theorem}
\begin{proof} From the definition we have
\begin{eqnarray*}
Q(X, X_0) & = & Q(X, X_1)\\
\vec{X_0}^2 - 2\vec{X_0}\cdot\vec{X} + \vec{X}^2 & = & \vec{X_1}^2 - 2\vec{X_1}\cdot\vec{X} + \vec{X}^2\\
2(\vec{X_0} - \vec{X_1})\cdot\vec{X} + \vec{X_1}^2 - \vec{X_0}^2 & = & 0.
\end{eqnarray*}
\end{proof}

We can now obtain a result which is inspired by Euclidean geometry.
In Euclidean geometry, the points on the perpendicular bisector of two points are equidistant from the two points.
In planar universal geometry we have the following generalised version of this theorem.

\begin{theorem}Given two points $X_0$ and $X_1$, the equiquadrance line is perpendicular to the line between the two points.
\end{theorem}
\begin{proof}
If $l_1$ is the line through $X_0$ and $X_1$, and $l_2$ is their perpendicular bisector then the metric dot product of the two lines is
\begin{eqnarray*}
\vec{l_1}\cdot\vec{l_2} & = & \begin{pmatrix}-b_2 \\ a_2\end{pmatrix}\begin{pmatrix}a & b \\ b & c\end{pmatrix}\begin{pmatrix}-b_1 \\ a_1\end{pmatrix}\\
                        & = & 2\begin{pmatrix}-(b(x_0 - x_1) + c(y_0 - y_1)) \\ a(x_0 - x_1) + b(y_0 - y_1)\end{pmatrix}\begin{pmatrix}a & b \\ b & c\end{pmatrix}\begin{pmatrix}x_1 - x_0 \\ y_1 - y_0\end{pmatrix}\\
                        & = & 2\begin{pmatrix}-(b(x_0 - x_1) + c(y_0 - y_1)) \\ a(x_0 - x_1) + b(y_0 - y_1)\end{pmatrix}\begin{pmatrix}a(x_1 - x_0) + b(y_1 - y_0) \\ b(x_1 - x_0) + c(y_1 - y_0)\end{pmatrix}\\
 & = & 0.
\end{eqnarray*}
Since their dot product is zero, the lines are perpendicular.
\end{proof}

\section{Altitudes}

Another idea we can borrow from Euclidean geometry is that of an altitude.
\begin{definition}In a geometry with metric $G$, given a point $X_0$ and a line $l_1$, an \emph{altitude} is a line which passes through $X_0$ and is perpendicular to $l_1$ with respect to the metric.
\end{definition}
While the definition of an altitude in planar universal geometry is equivalent to that in Euclidean geometry, the equation of the altitude line and its foot depend on the metric.
\begin{theorem}Given a point $X_0$ and a line $l_1$ in a geometry with metric $G$, the altitude $l_a$ has the form $\langle ba_1 - ab_1\!:\!ca_1 - bb_1\!:\!(ab_1 - ba_1)x_0 + (bb_1 - ca_1)y_0 \rangle$.
\end{theorem}
\begin{proof}
We require $l_1$ and $l_a$ to be perpendicular, which means we must have $\vec{l_1}\cdot_G\vec{l_a} = 0$.
This leads to the equation
\begin{eqnarray}
-(ba_1 - ab_1)b_a + (ca_1 - bb_1)a_a & = & 0.\nonumber
\end{eqnarray}
A solution to this equation is $\vec{l_a} = (bb_1 - ca_1, ba_1 - ab_1)^T$, giving $l_a = \langle ba_1 - ab_1\!:ca_1 - bb_1\! \!:\! c_a \rangle$.
We require the altitude $l_a$ to pass through $X_0$ and so
\begin{eqnarray*}
l_a(X_0) & = & 0\\
(ba_1 - ab_1)x_0 + (ca_1 - bb_1)y_0 + c_1 & = & 0\\
c_1 & = & (ab_1 - ba_1)x_0 + (bb_1 - ca_1)y_0.
\end{eqnarray*}
\end{proof}

\begin{definition}In a geometry with metric G, given a line, a point and the altitude from the point to the line, the \emph{altitude foot} is defined as the point where the altitude intersects the line.
\end{definition}
\begin{theorem}\label{theorem:foot}Given a point $X_0$ and a line $l_1$ in a geometry with metric $G$, the foot of the altitude is $F = [x_F, y_F]$ where
\begin{eqnarray}
F_x & = & -\frac{b_1x_0(ba_1 - ab_1) + (ca_1 - bb_1)(c_1 + b_1y_0)}{\|\vec{l_1}\|}\\
F_y & = &  \frac{a_1y_0(ca_1 - bb_1) + (ba_1 - ab_1)(c_1 + a_1x_0)}{\|\vec{l_1}\|}.
\end{eqnarray}
\end{theorem}
\begin{proof}
For a given $l_1$ and $X_0$, we wish to find the foot of the altitude, $F = [x_F, y_F]$, which is the point where $l_1$ intersects the altitude $l_a$. From Theorem \ref{th:intersect}, which gives the intersection of two lines, we find that
\begin{eqnarray}
F_x & = & \frac{b_1c_a - b_ac_1}{a_1b_a - a_ab_1}\nonumber\\
    & = & \frac{-b_1((ba_1 - ab_1)x_0 + (ca_1 - bb_1)y_0) - (ca_1 - bb_1)c_1}{a_1(ca_1 - bb_1) - (ba_1 - ab_1)b_1}\nonumber\\
    & = & \frac{-b_1x_0(ba_1 - ab_1) - (ca_1 - bb_1)(c_1 + b_1y_0) }{\|\vec{l_1}\|}\nonumber\\
F_y & = & \frac{a_ac_1 - a_1c_a}{a_1b_a - a_ab_1}\nonumber\\
    & = & \frac{c_1(ba_1 - ab_1) - a_1((ab_1 - ba_1)x_0 + (bb_1 - ca_1)y_0)}{\|\vec{l_1}\|}\nonumber\\
    & = & \frac{(ba_1 - ab_1)(c_1 + a_1x_0) + (ca_1 - bb_1)a_1y_0}{\|\vec{l_1}\|}.\nonumber
\end{eqnarray}
\end{proof}
While the equation for the foot of the altitude is reasonably complex, the quadrance between the foot and the original point is relatively simple.
\begin{definition}The \emph{point-line quadrance} between a point and a line is defined as the quadrance between the point and the foot of the altitude formed with the line.
\end{definition}
\begin{theorem}Given a point $X_0$ and a line $l_1$ in a geometry with metric $G$, the point-line quadrance is
\begin{eqnarray}
Q(X_0, l_1) & = & \frac{l_1(X_0)^2}{\|\vec{l_1}\|}\Delta_G.\label{eq:point-line}
\end{eqnarray}
\end{theorem}
\begin{proof}
If $X_0 = [x_0, y_0]$ and the foot of the altitude is $F = [x_F, y_F]$ then from the definition we have
\begin{eqnarray}
Q(X_0, l_1) & = & Q(X_0, F)\nonumber\\
        & = & a(x_F - x_0)^2 + 2b(x_F - x_0)(y_F - y_0) + c(y_F - y_0)^2.\label{eq:foot}
\end{eqnarray}
From Theorem \ref{theorem:foot} we have
\begin{eqnarray}
x_F - x_0 & = & \frac{-b_1x_0(ba_1 - ab_1) - (ca_1 - bb_1)(c_1 + b_1y_0) - (ab_1^2 - 2ba_1b_1 + ca_1^2)x_0}{ab_1^2 - 2ba_1b_1 + ca_1^2}\nonumber\\
          & = & \frac{ba_1b_1x_0 - (ca_1 - bb_1)(c_1 + b_1y_0) - ca_1^2x_0}{\|\vec{l_1}\|}\nonumber\\
          & = & \frac{(bb_1 - ca_1)(a_1x_0 + b_1y_0 + c_1)}{\|\vec{l_1}\|}\nonumber\\
          & = & \frac{l_1(X_0)}{\|\vec{l_1}\|}(bb_1 - ca_1)\nonumber\\
y_F - y_0 & = & \frac{(ba_1 - ab_1)(c_1 + a_1x_0) + (ca_1 - bb_1)a_1y_0 - (ab_1^2 - 2ba_1b_1 + ca_1^2)y_0 }{ab_1^2 - 2ba_1b_1 + ca_1^2}\nonumber\\
          & = & \frac{(ba_1 - ab_1)(c_1 + a_1x_0) + ba_1b_1y_0 - ab_1^2y_0 }{\|\vec{l_1}\|}\nonumber\\
          & = & \frac{(ba_1 - ab_1)(a_1x_0 + b_1y_0 + c_1)}{\|\vec{l_1}\|}\nonumber\\
          & = & \frac{l_1(X_0)}{\|\vec{l_1}\|}(ba_1 - ab_1).\nonumber
\end{eqnarray}
Combining these expressions with equation \ref{eq:foot} we finally find
\begin{eqnarray}
Q(X_0, l_1) & = & \left(\frac{l_1(X_0)}{\|\vec{l_1}\|}\right)^2\left( a(bb_1 - ca_1)^2 + 2b(bb_1 - ca_1)(ba_1 - ab_1) + c(ba_1 - ab_1)^2 \right)\nonumber\\
        & = & \left(\frac{l_1(X_0)}{\|\vec{l_1}\|}\right)^2\left( (ac^2 - b^2c)a_1^2 - 2(abc - b^3)a_1b_1 + (a^2c - ab^2)b_1^2 \right)\nonumber\\
        & = & \left(\frac{l_1(X_0)}{\|\vec{l_1}\|}\right)^2(ac - b^2)( ca_1^2 - 2ba_1b_1 + ab_1^2)\nonumber\\
        & = & \frac{l_1(X_0)^2}{\|\vec{l_1}\|}\Delta_G.
\end{eqnarray}
\end{proof}

The results in this chapter show how ideas from Euclidean geometry can be expressed in planar universal geometry.
In the following chapter we will use these ideas to build up a theory of conics.
