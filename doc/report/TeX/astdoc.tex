% Copyright (C) 1990 Sake J. Hogeveen.

% This file contains the LaTeX source of the documentation for the BibTeX
% style file ASTRON.BST and the complementary LaTeX style ASTRON.STY.

% Copyright Notice:
% The files ASTRON.BST, ASTRON.STY, ASTDOC.TEX, ASTDOC.BIB, EXAMPLE.TEX,
% EXAMPLE.BIB, MNEMONIC.BIB, and TEMPLATE.BIB are a package.
% You may copy and distribute them freely for non-commercial purposes,
% provided that you keep the package together and this copyright notice
% in tact.
% You may not alter or modify the files; this helps to ensure that all
% distributions of ASTRON.BST and related files are the same.
% If you make any modifications, then you must give the files new names,
% other than the present.
% The author bears no responsibilities for errors in this document or
% the software it describes; and shall not be held liable for any indirect,
% incidental, or consequential damages.

% If you find any bugs in the style files, or omissions in the documentation,
% please report them to: E-mail A410SAKE@SARA.NL; UUCP: l42@nikhefh.nikhef.nl

\documentstyle[titlepage,astron]{article}

\textwidth 5.5in
\textheight 8.5in
\hoffset -.35in
\voffset -.6in

\setcounter{secnumdepth}{2}
\setcounter{tocdepth}{3}

\def\BibTeX{{\rm B\kern-.05em{\sc i\kern-.025em b}\kern-.08em\TeX }}

\title{%{\huge\tt astron.bst} \bigskip \\
        A \BibTeX\ Style for Astronomical Journals}
\author{(for use with BibTeX 0.99c) \bigskip \\
 Sake J. Hogeveen}
\date{\vspace{.5in}
 \fbox{\parbox{3in}{\small This is a preliminary version. Please
 report any bugs in the style files, and errors or omissions in the
 documentation to one of the E-mail addresses below.
 This package is sent to several astronomical journals, with a request
 for their official approval of its use.
 Version 1.0 will hopefully contain a list of journals that have given
 their consent.}}\vfil
 \parbox{4.5 in}{\normalsize Copyright \copyright 1990, Sake J. Hogeveen.
 The files {\tt astron.bst}, {\tt astron.sty}, {\tt astdoc.tex},
 {\tt astdoc.bib}, {\tt mnemonic.bib}, {\tt example.bib}, {\tt example.tex},
 and {\tt template.bib} are a package.
 You may copy and distribute them freely for non-commercial purposes,
 provided that you keep the package together and this copyright notice
 in tact.
 You may not alter or modify the files; this helps to ensure that all
 distributions of {\tt astron.bst} and related files are the same.
 If you make any modifications, then you must give the files new names,
 other than the present. \\
 The author bears no responsibilities for errors in this document or
 the software it describes; and shall not be held liable for any indirect,
 incidental, or consequential damages.}\vfil
 \parbox{5.5 in}{\begin{center} \normalsize
 Astronomical Institute `Anton Pannekoek', Roetersstraat 15,
 1018 {\sc wb} Amsterdam, \\ The Netherlands \medskip \\
 E-mail: Earn/Internet: {\tt A410SAKE@SARA.NL};
 UUCP: {\tt l42@nikhefh.nikhef.nl}\end{center}}}

\begin{document}
\maketitle
\renewcommand{\BibTeX}{BibTeX}
\renewcommand{\LaTeX}{LaTeX}
\tableofcontents
\newpage

\section*{Introduction}
\addcontentsline{toc}{section}{Introduction}

This document describes the \BibTeX\ style {\tt astron.bst}.
\BibTeX\ is a programme that makes it much easier to produce a list of
references for papers that are prepared with \LaTeX, and it comes with
nearly every distribution of \LaTeX.
Consult your local \TeX{}nician if you cannot find it on the computer(s)
you use.

The {\tt astron.bst} style is designed to produce a list of references in
the author-year format that is commonly used among astronomical journals.
The style file produces references in the exact format of
{\sl Astronomy and Astrophysics}, the European astronomical journal.
However, the formats of the various astronomical journals (notably
Monthly Notices of the Royal Astronomical Society, Astrophysics and
Space Science, The Astronomical Journal, and The Astrophysical Journal)
only differ in typographical details, such as punctuation and the use
of italics.

In the near future we may expect astronomical journals to accept
source or {\sc dvi} files of papers prepared with \TeX\ and \LaTeX.
It will be possible to use \BibTeX\ and the style {\tt astron.bst}
presented here {\sl without restrictions\/} with any of the \LaTeX\ style
files that may be distributed by the journals or their publishers.

For some aspects of scientific publishing with \TeX\ and \LaTeX, see
appendix~\ref{aspects}

\section{\BibTeX}

This documentation is not intended as an introduction to \BibTeX,
nor to \LaTeX.
You are assumed to be familiar with both.
If you are not: \LaTeX\ is documented in its {\sl User's Guide and
Reference Manual\/} by Leslie Lamport \cite*{lamport}.
\BibTeX\ is documented by its author: Oren Patashnik \cite*{patashnik}.
The manual comes with each distribution of \BibTeX.
It explains how \BibTeX\ should be used, and how style files can be created
or adapted.

\BibTeX\ is also documented in section 4.2.3, `Using \BibTeX', and appendix~B,
`The Bibliography Database', of the \LaTeX\ {\sl User's Guide}.
If you only want to use existing bibliography styles, the
information in the \LaTeX\ {\sl User's Guide and Reference Manual\/}
\cite{lamport} is sufficient.

\section{The `{\tt astron}' style files}

With this package you should find the \BibTeX\ style file {\tt astron.bst}
and the complementary \LaTeX\ style file {\tt astron.sty}.
You should install both on the computer on which you run \LaTeX\ and
\BibTeX.
Preferably you should put the files in the directory where the standard
and/or optional \LaTeX\ and \BibTeX\ style files are kept.
If you are on a multi-user system and have no permission to write
to these directories, ask your system manager to install them for you.
If you are the only astronomer on the system, and hence the only user of
the `{\tt astron}' style, you may also put the files in your working
directory.

\subsection{$\backslash${\tt cite} and $\backslash${\tt cite*}}

Citations in the author-year format usually look like: `(Author, year)'.
But sometimes the name of the author is part of the running text,
and you want to make a citation look like: `\ldots{} Author (year)'.
To allow for both forms of citations, {\tt astron.sty} provides two
commands: \verb|\cite| and \verb|\cite*|.
The command \verb|\cite{|{\it label\/}\verb|}| will produce the full
citation `(Author, year)', while the command
\verb|\cite*{|{\it label\/}\verb|}| will produce the short citation `(year)'.

These commands are not standard \LaTeX\ (\LaTeX\ only provides the \verb|\cite|
command).
They are added to the vocabulary of any \LaTeX\ document style by including
the `astron' option in your document style call, e.g.:
\begin{quotation}
 \noindent\verb|\documentstyle[11pt,astron]{article}|
\end{quotation}
where `{\tt 11pt}' and `{\tt article}' represent the document styles you
chose to use.

If you want to make a citation of an other format than the formats produced
by \verb|\cite| or \verb|\cite*|, you will have to type the entire citation
yourself and use the \verb|\nocite| command.
The call `\verb|\nocite{|{\it label\/}\verb|}|' in your document will cause
\LaTeX\ and \BibTeX\ to include the citation indicated by {\it label\/} in
the list of references, but the \verb|\nocite| command will {\sl not\/}
be replaced by the {\it key\/} text.

\subsection{{\tt astron.bst} and {\tt astron.sty}}

The style file {\tt astron.bst} causes \BibTeX\ to produce a
{\it document\/}{\tt .bbl} file with a list of references according
to the usual format:
%
\begin{quotation}
\noindent \verb|\begin{thebibliography}{}| \\
\verb|\bibitem[|{\it key}\verb|]{|{\it label}\verb|}|{\it bibliography entry} \\
\indent $\vdots$ \\
\verb|\end{thebibliography}|
\end{quotation}
%
The {\it label\/} is, of course, the label which you use in the
\verb|\cite{|{\it label\/}\verb|}| and \verb|\cite*{|{\it label\/}\verb|}|
calls in your \LaTeX\ document, and by which \BibTeX\ recognizes entries
in the {\it database\/}{\tt.bib} file.

\noindent The style file {\tt astron.bst} causes \BibTeX\ to produce {\it key\/}s
with the format: \verb|{|{\it Author\/}\verb|}{|{\it year\/}\verb|}|.
The style file {\tt astron.sty} causes \LaTeX\ to replace a
\verb|\cite{|{\it label\/}\verb|}| call by the text `(Author, year)',
and a \verb|\cite*{|{\it label\/}\verb|}| call by the text `(year)'.
Due to the definition of \verb|\@biblabel| in {\tt astron.sty}, the {\it key\/}
is not printed in the actual list of references.

\subsection{Required, optional, and ignored fields}
\label{fields}

The bibliography style `{\tt astron}' has its own ideas about which fields
in an entry are required, optional or ignored.
Astronomical journals make an effort of `economic' printing, which means
that they leave everything out that is not absolutely necessary.

The classes of the fields in each entry are tabulated in
appendix~\ref{classes}.
One thing I will give away here: in the `author-year' citation system
it is obvious that {\tt author} and {\tt year} information is
required for every entry.
Remember that `required' means that \BibTeX\ issues a warning if the field
is empty, i.e., if the information is not available in your
{\it database\/}{\tt .bib} file.

\section{Examples}

The files {\tt example.bib} and {\tt example.tex} show what a bibliographic
database file might look like, and demonstrate the use of the \verb|\cite|
and \verb|\cite*| commands in various forms.

\section{Abbreviations}
\label{abbreviations}

The file {\tt mnemonic.bib} contains a list of mnemonics which may
be used in the `{\tt journal}' fields of the entries in the bibliographic
database.
If you want to use the mnemonics, you should include {\tt mnemonic.bib}
into the \verb|\bibliography| call:
%
\begin{quotation}
\noindent\verb|\bibliography{mnemonic,|{\it other bibliographic
	database files\/}\verb|}|
\end{quotation}
%
If you have an entry: `{\tt journal = aa}' (note: {\sl without\/}
quotes or braces around `{\tt aa}'), then \BibTeX\ will cause the
mnemonic to be replaced by the text {\it Astron.\ Astrophys.\/} in the
final list of references.

The list in appendix~\ref{mnemonics} contains the same journals and
periodicals as section 001 `Periodicals' in
{\sl Astronomy and Astrophysics Abstracts}, Vol.~49A \cite{burkhardt}.
The abbreviations follow the recommendations of the ``International
List of Periodical Title Word Abbreviations'', so you are strongly
encouraged to use them.

If you want to cite periodicals which are not in {\tt mnemonic.bib},
or if you think you will never use 90\% of the mnemonics in the
supplied file, you could create your own mnemonics file, and copy
from {\tt mnemonic.bib} the abbreviations which you do use.

\section{Maintaining the database}

To setup and maintain bibliographic databases for \BibTeX, this package
contains a file {\tt template.bib}, which contains templates of the
entries with their required and optional fields (according to
{\tt astron.bst}).
If you want to add an entry to a database, you can copy the relevant
template from {\tt template.bib} into your database file, and then
complete the entry with the relevant information.

I have used this method to maintain the databases for quite some time,
and I find that it saves me from a lot of arduous typing, and many
typing errors.

\section{Credits}

The \BibTeX\ and \LaTeX\ `{\tt astron}' styles are adaptations of the
`{\tt apalike}' styles developed by Oren Patashnik and Suzan King.
The \verb|\cite| and \verb|\cite*| trickery is adapted from
`{\tt named.bst}' by Peter~F. Patel-Schneider.

\bibliographystyle{astron}
\bibliography{mnemonic,astdoc}

\appendix

\newpage
\section{Aspects of publishing with \TeX\ and \LaTeX}
\label{aspects}

New developments, such as \TeX\ and \LaTeX, are about to radically change
the conventional ways of scientific publishing.
Experiments with the acceptance of papers prepared with \TeX\ or
\LaTeX\ are already going on \cite{daniel}.

The principle is simple and appealing.
Authors prepare their papers with \TeX\ or \LaTeX, and send their manuscripts
to the publishers on floppy disk or via electronic mail.
A publisher runs \TeX\ or \LaTeX\ on the manuscript, and produces a
(high quality) printable copy of the paper.
The costly and time consuming steps of typesetting and proofreading
would then be eliminated.

But before everything will run smoothly, some aspects of scientific publishing
with \TeX\ and \LaTeX\ will have to be considered, and some problems need to
be addressed.

\subsection{Generalized Mark-up}

Although papers in the various scientific journals exhibit a great variety
in typographical appearance, a common underlying structure can be recognized.
The papers can be characterized by: a heading (with the title of the paper,
and the author's names and affiliations); an abstract or summary; sections;
tables; figures; equations; a list of references; and sometimes appendices.
The typographical form of a paper is intended to support this structure.
The typography is, in fact, derived from the structure, as becomes evident
from the typographical instructions that are scribbled on to the oldfashioned
typewriter written manuscripts by a desk editor.

A desk editor recognizes the structural elements of a paper from the
typography of the typewriter written manuscript.
The author has {\sl implicitly\/} structured his manuscript by applying
some rudimentary typography.
The structure of a manuscript can be formalized, by asking the author to
{\sl explicitly\/} state what comes next, e.g. like this:
%
\begin{quotation}
\noindent\begin{verbatim}
title:       Gnats and Gnus again
author:      A.N. Author
affiliation: The Gnats and Gnus Institute,
             Somewhere Lane 15, Downtown, Inthecountry.
abstract:
 Some people get fed up with examples that are about Gnats and Gnus.
 This paper brings about new information, which shows that Gnats and
 Gnus are a species with such unexpected virtues, that the reader will
 never be agonized again, but, to the contrary, cannot wait to learn
 more about them.
introduction:
\end{verbatim}
\par$\vdots$
\end{quotation}
%
A desk editor could again scribble his remarks in this manuscript,
to indicate to a typesetter what typographical form is required for
each of the structural elements.
The `scribbeling' of the desk editor is, by the way, called `mark-up'.
An other desk editor, of a different journal, could also scribble
in {\sl his\/} remarks, which would make sure that the typesetter
applies the typographical rules of this other journal.

So we have not only {\sl formalized\/} the mark-up of the paper, we
have also made it independent of the typographical form.
Mark-up which is independent of the typographical form is said to be
{\sl generalized\/} mark-up.

The idea of modern (scientific) publishing is that the author supplies
a manuscript, which is marked-up employing a `Generalized Mark-up
Language'\footnote{
 In fact, there is at present an international effort being made of
 the development of what is called the `Standard Generalized Mark-up
 Language' (SGML).
 SGML will be~/~is an ISO standard.
 SGML will be mainly used by publishers and by organisations to maintain
 large databases.

 As an author you do not have to worry that you will yet again have
 to learn something new: suppliers of word-processors will incorporate
 SGML in their products, so you will produce SGML manuscripts without
 knowing it.}.

This manuscript can be processed by {\em any\/} publisher into the
typographical format of a particular publication.
The advantages of the concept are obvious: 1.~the author does not
have to bother with typographical matters, and 2.~the publisher is
completely in control of the appearance of a publication, just as he
always has been.

\subsection{\TeX}

So how do \TeX\ and \LaTeX\ fit into this concept?
Let us first look at \TeX, and then at \LaTeX.

\TeX\ is a computerized typesetting system, geared towards the typesetting
of manuscripts which contain a lot of mathematics.
Like the traditional typesetting systems, \TeX\ requires information
from the user about {\sl where\/} and {\sl how\/} things are to be put on
paper.
\TeX\ provides many basic commands, but these commands are hardly ever
used directly.
The commands are grouped in {\sl macros}, which perform (typographically)
logical functions.
Macros are grouped into {\sl macro packages}.
\TeX\ comes with a standard macro package `{\tt plain.tex}', and it is this
macro package that is used by most authors who say they use \TeX.

However, this `plain' macro package still requires information from the
user about where and how things are to be put on paper.
And it is here that a problem arises.
Because if `plain' \TeX\ were to be used for scientific publishing, it
would mean that authors have to concern themselves with the layout and
the typographical details of the journal(s) they are publishing in.
Although publishers would not dare to doubt the scientific capabilities
of authors, they have serious concerns about their qualities as typographers.
And with good reason, as years of experience with `camera ready' manuscripts
have shown.

But, because \TeX\ actually is a computer programme, there are great powers
hidden inside.
By building on the `plain' macros, or by developing completely new ones, the
concept of macros can be carried further, until a package is achieved which,
to the user, looks like a `Generalized Mark-up Language'.

\subsection{\LaTeX\ as a Generalized Mark-up Language}

A \TeX\ macro package with the properties of a Generalized Mark-up Language
has already been developed.
It is called \LaTeX.
With \LaTeX\ the author uses `generalized mark-up' commands such as
`\verb|\title|', `\verb|\begin{abstract}|', `\verb|\section|',
`\verb|\appendix|', which bear no relation to a particular typographical
format at all.
The typographical format of the final printed copy of a \LaTeX\ manuscript
is determined by so-called `style files'.
\LaTeX\ comes with four standard style files, which allow the user to
produce typographically sound articles (papers), books, reports, and letters.
But these style files may be replaced by others, to produce an entirely
different typographical format of a document from the {\sl same\/}
mark-up commands.
Thus, if a publisher accepts the source text of a \LaTeX\ manuscript, he
can convert it into printed output with almost any desired 
typographical format,
{\sl without\/} changing the original text.

\subsection{\TeX, and the `typewriter syndrome'}

If you are an editor or a publisher, you will have met authors who
say that they much prefer \TeX\ over \LaTeX.
They try to convince you that they can do much more with it than can be
done with \LaTeX.
What they mean is, that they have more direct control of what the
{\sl printed output\/} looks like, and hence, over the typography of their
product.
We should realize that authors who say they use \TeX, in fact use the
`plain' macro package.
Some 90\% of these authors use `plain' \TeX\ as a sophisticated typewriter.
For instance, when they start a new section, they tell \TeX\ to leave some
white space between the preceding text and the new section heading,
switch to a different font, type the heading, leave some white space again,
switch back to the font for the running text, and continue to type the
next paragraph.

This is all fine when an author types a manuscript for his own purposes,
or one that will be typeset again by a professional typesetter.
Of course, a publisher could provide a set of instructions which tell
the author how everything should be formatted, just like the instructions
that are provided for `camera ready' papers.
But publishers know how well authors keep to these instructions, and many
authors have nightmares about camera ready manuscripts, which cost them
blood, sweat, tears, and far too much time to prepare.

Furthermore, \TeX\ is not intended to be used in this way.
When `plain' \TeX\ is used as a typewriter, not even 5\% of its potential
capabilities are challenged.
The design of \TeX\ allows it to be adapted to the structural and typographical
demands of the publications of many, different branches of science, by
means of the `macro' concept.
For very specialized branches, special macro packages can be developed
(and are developed, such as the $\cal AMS$\TeX\ package of the American
Mathematical Society).
Other branches, like astronomy, could do with a `general purpose' package,
such as \LaTeX.

Authors will have to adapt to the concept of Generalized Mark-up, and
be cured from the `typewriter syndrome'.
Only through generalized mark-up, the principle of directly publishing
the manuscript provided by the author will work.
And the benefits will be great:
it allows the author to fully concentrate on the writing of the manuscript,
without worries about the layout; it leaves the publisher in full
control of the typographical appearance of his publications; and the
time between submission and publication can be reduced considerably.

\subsection{Consultation and education}

Like with every beautiful concept, there are some practical hazards and
problems involved in scientific publishing with \TeX\ and \LaTeX.
The mark-up commands provided by \LaTeX\ do not cover every peculiarity
you may have in scientific papers, and not all commands are strictly
`generalized', i.e., some do have a direct typographical effect\footnote{
 This is not a problem peculiar to \LaTeX, it is a problem that designers
 of generalized mark-up languages are struggling with in general.}.
%
And then authors should use the proper commands for their mark-up, which
they have to select from an overwhelming number of available commands.

Publishers can provide special or optional style files, which add new
mark-up commands to the \LaTeX\ vocabulary.
But this has to be done very cautiously, because every new command means a
deviation from standard \LaTeX, and requires extra attention from the
authors.
New commands should not be added without proper consultation of the
community of authors that has to use a particular macro package.

Authors will have to be educated about the proper mark-up of their papers.
Experience at Springer-Verlag (Heidelberg) and Kluwer Academic Publishers
(Dordrecht), with \TeX\ and \LaTeX\ manuscripts provided by authors, has
shown that the much greater possibilities of these systems in comparison to
the old typewriter, are only to often used in the wrong way.
This means that \TeX\ and \LaTeX\ manuscripts need so much polishing and
brushing up, that the effort involved equals or sometimes surpasses that
of having the manuscript typeset in the traditional way.

What authors have to learn about the mark-up of their papers, are rules
that generally apply to the proper typesetting of mathematical texts.
These rules are not dependent of a particular journal (although, of course,
some branches of science may have their peculiarities, but these are then
again common for the branch).
Rules of proper typesetting are mentioned at many places in the
{\sl TeX{}book\/} and in the {\sl \LaTeX\ User's Guide and Reference Manual}.
But the learning of \TeX\ and \LaTeX\ is at present still a matter
of teaching it yourself, and many authors seem to miss the good advice
that is given.

\subsection{Concluding remarks}

The use of \TeX\ and \LaTeX\ for scientific publishing will only succeed
if we adapt to the concept of `generalized mark-up'.
Only in that way an author can fully concentrate on the writing of his
manuscript, and leave the typographical intricacies to the publisher.
Much has yet to be learned by everyone involved: authors, editors, and
publishers.
We can learn from each other: publishers can learn from authors what
is wrong about their macro packages; authors can learn from publishers
which rules of proper typesetting they violate.

It would be a good thing if the processes of consultation and education
were formalized in working groups, or something similar.
I know there are publishers who are reluctant about formalized consultation,
because they are afraid it slows down their efforts to get going with
\TeX\ and \LaTeX.
However, it is my opinion that they can only benefit from participating
in some form of formalized consultation.
The introduction of \TeX\ and \LaTeX\ will take time anyway, and by listening
to the authors, publishers may be prevented from releasing macro packages
which are not accepted by the authors.

Authors could in such negotiations stress that they will only opt for some
form of generalized mark-up, because working truly camera ready would
take too much of their time.
An inventory could be made of what authors have to learn about the proper
mark-up of mathematical texts, and courses could be set up to educate the
new generation of authors.
I think that some time from now universities will provide introductory courses
in \LaTeX, just as they are now providing introductory courses in computer
programming. \smallskip \\

In the mean time, \LaTeX\ users can enjoy this \BibTeX\ style, which extends
the concept of generalized mark-up to such a horrifying thing as the
compilation of a list of references.



\newpage
\section{Classes of the fields}
\label{classes}

Fields in the entries of the bibliographic database are attributed one of
three classes: required, optional, or ignored.
The table in this appendix lists the class of each field in each entry.
See also Sect.~\ref{fields}.

\newcommand{\R}{{\sl R}}
\renewcommand{\O}{{\sl O}}
\newcommand{\I}{--}

\vfil
\noindent\begin{center}
{\bf Table \thesection.1:} Classes of the fields. \\
(\R $=$ Required, \O $=$ Optional, \I $=$ Ignored) \medskip \\
%
\begin{tabular}{lccccccc} \hline
 & \multicolumn{7}{c}{\bf Entry} \\
{\bf Field} & article & book & booklet & conference$^a$ & inbook & incoll.$^b$
 & inproc.$^c$ \\ \hline\hline
address      & \I & \O & \O & \O & \O & \O & \O \\
annote       & \I & \I & \I & \I & \I & \I & \I \\
author       & \R & \R & \R & \R & \R & \R & \R \\
booktitle    & \I & \I & \I & \R & \I & \R & \R \\
chapter      & \I & \I & \I & \I & \R & \O & \I \\
crossref     & \O & \O & \I & \O & \O & \O & \O \\
edition      & \I & \O & \I & \I & \O & \O & \I \\
editor       & \I & \O & \I & \O & \O & \O & \O \\
howpublished & \I & \I & \O & \I & \I & \I & \I \\
institution  & \I & \I & \I & \I & \I & \I & \I \\
journal      & \R & \I & \I & \I & \I & \I & \I \\
key          & \O & \O & \O & \O & \O & \O & \O \\
month        & \I & \I & \I & \I & \I & \I & \I \\
note         & \O & \O & \O & \O & \O & \O & \O \\
number       & \I & \O & \I & \O & \O & \O & \O \\
organisation & \I & \I & \I & \O & \I & \I & \O \\
pages        & \O & \I & \I & \O & \O & \O & \O \\
publisher    & \I & \R & \I & \O & \R & \R & \O \\
school       & \I & \I & \I & \I & \I & \I & \I \\
series       & \I & \O & \I & \O & \O & \O & \O \\
title        & \I & \R & \R & \I & \R & \I & \I \\
type         & \I & \I & \I & \I & \I & \I & \I \\
volume       & \O & \O & \I & \O & \O & \O & \O \\
year         & \R & \R & \R & \R & \R & \R & \R \\ \hline
\end{tabular}
\end{center}
$^a$ `conference' is the same as `inproceedings' \\
$^b$ incoll.\ $=$ incollection \\
$^c$ inproc.\ $=$ inproceedings

\newpage

\null\vfil
\noindent\begin{center}
{\bf Table \thesection.1:} Classes of the fields ({\it continued\/}). \\
(\R $=$ Required, \O $=$ Optional, \I $=$ Ignored) \medskip \\
%
\begin{tabular}{lccccccc} \hline
 & \multicolumn{7}{c}{\bf Entry} \\
{\bf Field} & manual & masters.$^d$ & misc & phd.$^e$ & proc.$^f$ & tech.$^g$
 & unpubl.$^h$ \\ \hline\hline
address      & \O & \O & \I & \O & \O & \O & \I \\
annote       & \I & \I & \I & \I & \I & \I & \I \\
author       & \R & \R & \R & \R & \R & \R & \R \\
booktitle    & \I & \I & \I & \I & \I & \I & \I \\
chapter      & \I & \I & \I & \I & \I & \I & \I \\
crossref     & \I & \I & \I & \I & \I & \I & \I \\
edition      & \O & \I & \I & \I & \I & \I & \I \\
editor       & \I & \I & \I & \I & \O & \I & \I \\
howpublished & \I & \I & \O & \I & \I & \I & \I \\
institution  & \I & \I & \I & \I & \I & \O & \I \\
journal      & \I & \I & \I & \I & \I & \I & \I \\
key          & \O & \O & \O & \O & \O & \O & \O \\
month        & \I & \I & \I & \I & \I & \I & \I \\
note         & \O & \O & \O & \O & \O & \O & \R \\
number       & \I & \I & \I & \I & \O & \I & \I \\
organisation & \O & \I & \I & \I & \O & \I & \I \\
pages        & \I & \I & \I & \I & \I & \I & \I \\
publisher    & \I & \I & \I & \I & \O & \I & \I \\
school       & \I & \R & \I & \R & \I & \I & \I \\
series       & \I & \I & \I & \I & \O & \I & \I \\
title        & \R & \I & \O & \I & \R & \R & \R \\
type         & \O & \O & \O & \O & \O & \O & \O \\
volume       & \I & \I & \I & \I & \O & \I & \I \\
year         & \R & \R & \R & \R & \R & \R & \R \\ \hline
\end{tabular}
\end{center}
$^d$ masters.\ $=$ mastersthesis \\
$^e$ phd.\ $=$ phdthesis \\
$^f$ proc.\ $=$ proceedings \\
$^g$ tech.\ $=$ techreport \\
$^h$ unpubl.\ $=$ unpublished

\newpage
\section{Changing the style files yourself}

The `{\tt astron}' bibliography style has been designed to meet the
requirements of astronomical journals.
Maybe you are working on documents which could do with about the same
documentstyle, but not quite.
Then you may want to modify the `{\tt astron}' styles to your needs.
To help you on your way, here is a summary of what happens in each style
file.

You do remember to name modified style files anything other than
{\tt astron.bst} or {\tt astron.sty}, don't you?

\subsection{{\tt astron.bst}}

The style {\tt astron.bst} causes \BibTeX\ to create a bibliography file
which looks like:
%
\begin{quotation}
\noindent \verb|\begin{thebibliography}{}| \\
\verb|\bibitem[|{\it key}\verb|]{|{\it label}\verb|}|{\it bibliography entry} \\
\indent $\vdots$ \\
\verb|\end{thebibliography}|
\end{quotation}
%
The {\it key\/} is formatted from the {\tt author} and {\tt year} entries
in the bibliographic database.
It looks like: \verb|{|{\it Author\/}\verb|}{|{\it year\/}\verb|}|.

The {\it bibliography entry\/} is formatted from the required and optional
information in the fields of an entry, like {\tt author}, {\tt year},
{\tt title}, {\tt volume}, {\tt pages}, et cetera.
According to the definitions in {\tt astron.bst}, \BibTeX\ takes care of
the typographical intricacies of the {\it bibliography entry}, such
as punctuation, italics for journal names and booktitles, boldface for
journal volumes, etc.

As an example, here is what {\tt astron.bst} made \BibTeX\ make of
\verb|\cite{lamport}| in this document:
%
\begin{quotation}
\noindent\begin{verbatim}
\bibitem[\protect\astroncite{Lamport}{1986}]{lamport}
Lamport, L.: 1986,
\newblock {\em LaTeX, A Document Preparation System},
\newblock Addison-Wesley, Reading, Massachusetts
\end{verbatim}
\end{quotation}

\subsection{{\tt astron.sty}}

The style file {\tt astron.sty} is a \LaTeX\ style file.
It defines the \verb|\cite| and \verb|\cite*| commands such that
\verb|\cite{|{\it label\/}\verb|}| is replaced by `(Author, year)'
in the running text, and \verb|\cite*{|{\it label\/}\verb|}| by
`(year)'.

It also defines the {\tt thebibliography} environment.
It causes the text `{\bf References}' to be produced as the heading of
the list of references in `{\tt article}\,s', and `{\bf Bibliography}'
as the heading in `{\tt report}\,s' and `{\tt book}\,s'

The {\tt thebibliography} environment is defined in terms of the
\LaTeX\ `{\tt list}' environment, in which the items are preceded
by `\verb|\bibitem|'.
All parameters related to the `{\tt list}' environment, such as
`{\tt itemsep}', `{\tt leftmargin}', etc., are set at the
\verb|\begin{thebibliography}| call. \medskip \\
%
To summarize in brief: the \BibTeX\ style {\tt astron.bst} takes care
of the typographical details of each item in the bibliography;
the \LaTeX\ style {\tt astron.sty} takes care of the global 
typographical appearance of the bibliography.

\twocolumn
\section{Mnemonics}
\label{mnemonics}

This appendix lists the mnemonics by which the abbreviations of journal
names may be called in the {\tt journal} fields of the entries in the
bibliographic database.
See also Sect.~\ref{abbreviations}. \medskip \\

% This file contains the abbreviations for journal names.
% From: Astronomy and Astrophysics Abstracts, 1990, Vol. 49A.

\newcommand{\printba}[2]{\noindent#2 \dotfill\ {\tt #1}\par}
\frenchspacing
\footnotesize

\sloppy
\printba{aao}{AAO Newsl.}
\printba{aavso}{AAVSO Newsl.}
\printba{acta}{Acta Astron.}
\printba{aaguv}{Acta Astron. Geophys. Univ. Comenianae}
\printba{actastsin}{Acta Astron. Sin.}
\printba{actapsin}{Acta Astrophys. Sin.}
\printba{agg}{Acta Geod. Geophys.}
\printba{ags}{Acta Geophys. Sin.}
\printba{appsb}{Acta Phys. Pol., Ser. B}
\printba{atash}{Acta Tech. Acad. Sci. Hung.}
\printba{advp}{Adv. Phys.}
\printba{advsr}{Adv. Space Res.}
\printba{aip}{AIP Conf. Proc.}
\printba{aavsob}{Am. Assoc. Variable Star Obs. Bull.}
\printba{aavsoc}{Am. Assoc. Variable Star Obs. Circ.}
\printba{amjp}{Am. J. Phys.}
\printba{aatar}{Anglo--Aust. Telesc., Annu. Rep.}
\printba{ag}{Ann. Geophys.}
\printba{apl}{Ann. Phys. (Leipzig)}
\printba{apny}{Ann. Phys. (N.Y.)}
\printba{app}{Ann. Phys. (Paris)}
\printba{ao}{Appl. Opt.}
\printba{appl}{Appl. Phys. Lett.}
\printba{apb}{Appl. Phys., B}
\printba{as}{Appl. Spectrosc.}
\printba{ahes}{Arch. Hist. Exact Sci.}
\printba{archuk}{Archeoastronomy (U.K.)}
\printba{asbts}{Archenhold--Sternw. Berlin--Treptow, Sonderdr.}
\printba{asbtvs}{Archenhold--Sternw. Berlin--Treptow, Vortr. Schr.}
\printba{aonaicnl}{Arecibo Obs./NAIC, Newsl.}
\printba{artsat}{Artif. Satell.}
\printba{aiisao}{Astrofiz. Issled. Izv. Spets. Astrofiz. Obs.}
\printba{af}{Astrofizika}
\printba{aa}{Astron. Astrophys.}
\printba{aar}{Astron. Astrophys. Rev.}
\printba{aas}{Astron. Astrophys. Suppl. Ser.}
\printba{abco}{Astron. Bull. (Carter Obs.)}
\printba{agas}{Astron. Ges., Abstr. Ser.}
\printba{asther}{Astron. Her.}
\printba{aiubc}{Astron. Inst. Univ. Brno, Contrib.}
\printba{astj}{Astron. J.}
\printba{an}{Astron. Nachr.}
\printba{astnow}{Astron. Now}
\printba{astraum}{Astron. Raumfahrt}
\printba{astriha}{Astron. Rechen-Inst.Heidelb., Mitt., Ser. A}
\printba{astrihb}{Astron. Rechen-Inst.Heidelb., Mitt., Ser. B}
\printba{astschu}{Astron. Sch.}
\printba{asttid}{Astron. Tidsskr.}
\printba{asttsir}{Astron. Tsirk.}
\printba{astvest}{Astron. Vestn.}
\printba{astzh}{Astron. Zh.}
\printba{astuai}{Astronomia UAI}
\printba{astie}{Astronomie}
\printba{asty}{Astronomy}
\printba{apj}{Astrophys. J.}
\printba{apjl}{Astrophys. J., Lett.}
\printba{apjss}{Astrophys. J., Suppl. Ser.}
\printba{aplc}{Astrophys. Lett. Commmun.}
\printba{apss}{Astrophys. Space. Sci.}
\printba{ap}{Astrophysics}
\printba{aanlsor}{Atti Accad. Naz. Lincei, Ser. Ottava, Rend.}
\printba{ajast}{Aust. J. Astron.}
\printba{ajp}{Aust. J. Phys.}
\printba{bavm}{BAV Mitt.}
\printba{bavr}{BAV Rundbrief}
\printba{bbsag}{BBSAG Bull.}
\printba{bestarnl}{Be Star Newsl.}
\printba{bildw}{Bild Wiss.}
\printba{bipm}{BIPM Circ. T}
\printba{boaumkt}{Biul. Obs. Astron. Uniw. M. Kopernika Toruniu}
\printba{boe}{Bol. Obs. Ebro}
\printba{broa}{Bol. ROA}
\printba{blm}{Boundary-Layer Meteorol.}
\printba{boop}{Boyden Obs., Occas. Publ.}
\printba{bastac}{Br. Astron. Assoc. Circ.}
\printba{bjp}{Bulg. J. Phys.}
\printba{baas}{Bull. Am. Astron. Soc.}
\printba{baps}{Bull. Am. Phys. Soc.}
\printba{bafoev}{Bull. Assoc. Fr. Obs. Etoiles Variables}
\printba{bastic}{Bull. Astron. Inst. Czech.}
\printba{basi}{Bull. Astron. Soc. India}
\printba{bcao}{Bull. Crimean Astrophys. Obs.}
\printba{bgsi}{Bull. Geogr. Surv. Inst.}
\printba{bicds}{Bull. Inf. Cent. Donn\'ees Stellaires}
\printba{boab}{Bull. Obs. Astron. Belgr.}
\printba{crassgvs}{C. R. Acad. Sci., S\'er. G\'en., Vie Sci.}
\printba{crassii}{C. R. Acad. Sci., II}
\printba{cjp}{Can. J. Phys.}
\printba{cm}{Celest. Mech.}
\printba{cassos}{Cent. Astron. Sci. Spat., Obs. Sol.}
\printba{cent}{Centaurus}
\printba{cpl}{Chem. Phys. Lett.}
\printba{chinaa}{Chin. Astron. Astrophys.}
\printba{chinp}{Chin. Phys.}
\printba{chinpl}{Chin. Phys. Lett.}
\printba{ciel}{Ciel}
\printba{cielt}{Ciel Terre}
\printba{ccotl}{Circ. Czech. Obs. Time Lattiude}
\printba{cinf}{Circ. Inf.}
\printba{ctls}{Circ. Time Latitude Serv.}
\printba{cqg}{Classical Quantum Gravity}
\printba{cast}{Comments Astrophys.}
\printba{cnpp}{Comments Nucl. Part. Phys.}
\printba{cppcf}{Comments Plasma Phys. Controlled Fusion}
\printba{cfsua}{Commun. Fac. Sci. Univ. Ankara, Ser. A2, A3}
\printba{cko}{Commun. Konkoly Obs.}
\printba{cmp}{Commun. Math. Phys.}
\printba{culo}{Commun. Univ. Lond. Obs.}
\printba{comp}{Comput. Phys.}
\printba{compcom}{Comput. Phys. Commun.}
\printba{cp}{Contemp. Phys.}
\printba{caosp}{Contrib. Astron. Obs. Skalnat\'e Pleso}
\printba{cap}{Contrib. Atmos. Phys.}
\printba{cdaut}{Contrib. Dep. Astron., Univ. Tokyo}
\printba{clo}{Contrib. Lick Obs.}
\printba{cncopb}{Contrib. Nicholas Copernicus Obs. Planetarium Brno}
\printba{cnjro}{Contrib. Nizamiah Japal-Rangapur Obs.}
\printba{cpp}{Contrib. Plasma Phys.}
\printba{cvvo}{Contrib. Van Vleck Obs.}
\printba{cr}{Cosmic Res.}
\printba{cryo}{Cryogenics}
\printba{czjp}{Czech. J. Phys., Sect. B}
\printba{drho}{Data Rep. Hydrogr. Obs., Ser. Astron. Geod.}
\printba{dgkbawb}{Dtsch. Geod. Komm. Bayer. Akad. Wiss., Reihe B}
\printba{dgkbawc}{Dtsch. Geod. Komm. Bayer. Akad. Wiss., Reihe C}
\printba{dgkbawe}{Dtsch. Geod. Komm. Bayer. Akad. Wiss., Reihe E}
\printba{epsl}{Earth Planet. Sci. Lett.}
\printba{erb}{Earth Rotation Bull.}
\printba{emp}{Earth, Moon, Planets}
\printba{esrev}{Earth--Sci. Rev.}
\printba{esab}{ESA Bull.}
\printba{esaiuenl}{ESA IUE Newsl.}
\printba{esaj}{ESA J.}
\printba{esoar}{ESO Ann. Rep.}
\printba{ejp}{Eur. J. Phys.}
\printba{epl}{Europhys. Lett.}
\printba{epn}{Europhys. News}
\printba{expast}{Exp. Astron.}
\printba{fiz}{Fizika}
\printba{fortp}{Fortschr. Phys.}
\printba{fp}{Found. Phys.}
\printba{funcp}{Fundam. Cosmic Phys.}
\printba{gaab}{G.~A.A.B.}
\printba{gast}{G. Astron.}
\printba{gem}{Gemini}
\printba{grg}{Gen. Relativ. Gravitation}
\printba{gca}{Geochim. Cosmochim. Acta}
\printba{ga}{Geomagn. Aehron.}
\printba{gafd}{Geophys. Astrophys. Fluid Dyn.}
\printba{gj}{Geophys. J.}
\printba{grl}{Geophys. Res. Lett.}
\printba{geo}{Geophysics}
\printba{geos}{GEOS Circ.}
\printba{gbg}{Gerlands Beitr. Geophys.}
\printba{heav}{Heavens}
\printba{hpa}{Helv. Phys. Acta}
\printba{hepnp}{High Energ. Phys. Nucl. Phys.}
\printba{hob}{Hvar Obs. Bull.}
\printba{hi}{Hyperfine Interact.}
\printba{iappp}{I.A.P.P.P. Commun.}
\printba{iauc}{IAU Circ.}
\printba{iauib}{IAU Inf. Bull.}
\printba{icarus}{Icarus}
\printba{i3etm}{IEEE Trans. Magn.}
\printba{i3etns}{IEEE Trans. Nucl. Sci.}
\printba{i3etps}{IEEE Trans. Plasma Sci.}
\printba{iersbb}{IERS Bull. B}
\printba{iersba}{IERS Bull.--A}
\printba{ijpap}{Indian J. Pure Appl. Phys.}
\printba{ijrsp}{Indian J. Radio Space Phys.}
\printba{ibvs}{Inf. Bull. Variable Stars}
\printba{iaatub}{Inst. Astron. Astrophys. Tech. Univ. Berlin, Mitt.}
\printba{iag}{Inst. Astron. Geod., Univ. Madr., Publ.}
\printba{ia}{Inst. Astron., Univ. Camb., Annu. Rep.}
\printba{icq}{Int. Comet Q.}
\printba{ijimw}{Int. J. Infrared Millimeter Waves}
\printba{ijmp}{Int. J. Mod. Phys. A}
\printba{ijtp}{Int. J. Theor. Phys.}
\printba{ip}{Inverse Probl.}
\printba{iastj}{Ir. Astron. J.}
\printba{iris}{IRIS Bull. A}
\printba{iskz}{Issled. Solntsa Krasnykh Zvezd}
\printba{intsa}{Itogi Nauki Tekh., Ser. Astron.}
\printba{intsppk}{Itogi NAuki Tekh., Ser. Plazmennye Protsessy Kosmose}
\printba{iaoe}{Izv. Astron. Obs. Ehngel'gardt.}
\printba{ikao}{Izv. Krym. Astrofiz. Obs.}
\printba{ivuzr}{Izv. Vyssch. Uchebn. Zaved., Radiofiz.}
\printba{jasa}{J. Acoust. Soc. Am.}
\printba{jaavso}{J. Am. Assoc. Variable Star Obs.}
\printba{jam}{J. Appl. Meteorol.}
\printba{jap}{J. Appl. Phys.}
\printba{jastf}{J. Astron. Fr.}
\printba{japast}{J. Astrophys. Astron.}
\printba{jas}{J. Atmos. Sci.}
\printba{jatp}{J. Atmos. Terr. Phys.}
\printba{jbasta}{J. Br. Astron. Assoc.}
\printba{jbis}{J. Br. Interplanet. Soc.}
\printba{jcp}{J. Chem. Phys.}
\printba{jcomp}{J. Comp. Phys.}
\printba{jel}{J. Electrostat.}
\printba{jfseu}{J. Fac Sci., Ege Univ., Ser. A}
\printba{jfm}{J. Fluid Mech.}
\printba{jgr}{J. Geophys. Res.}
\printba{jhast}{J. Hist. Astron.}
\printba{jms}{J. Mater. Sci.}
\printba{jmp}{J. Math. Phys.}
\printba{jmo}{J. Mod. Opt.}
\printba{jop}{J. Opt. (Paris)}
\printba{josaa}{J. Opt. Soc. Am. A}
\printba{josab}{J. Opt. Soc. Am. B}
\printba{jp}{J. Phys.}
\printba{jpa}{J. Phys. A}
\printba{jpb}{J. Phys. B}
\printba{jpd}{J. Phys. D}
\printba{jpe}{J. Phys. E}
\printba{jpg}{J. Phys. G}
\printba{jpsj}{J. Phys. Soc. Jpn.}
\printba{jpp}{J. Plasma Phys.}
\printba{jqsrt}{J. Quant. Spectrosc. Radiat. Transfer}
\printba{jrasc}{J. R. Astron. Soc. Can.}
\printba{jsp}{J. Stat. Phys.}
\printba{jvst}{J. Vac. Sci. Technol., A}
\printba{jetpl}{JETP Lett.}
\printba{jhatd}{Johns Hopkins APL Tech. Dig.}
\printba{jjap1}{Jpn. J. Appl. Phys., Part 1}
\printba{jjap2}{Jpn. J. Appl. Phys., Part 2}
\printba{kai}{Kapteyn Astron. Inst., Annu. Rep.}
\printba{kexto}{Kexue Tongbao}
\printba{kfnt}{Kinematika Fiz. Nebesn. Tel}
\printba{kob}{Kodaikanal Obs. Bull.}
\printba{kt}{Komet. Tsirk.}
\printba{ki}{Kosm. Issled.}
\printba{koz}{Kozmos}
\printba{kpm}{KPM}
\printba{lo}{Laser Optoelektron.}
\printba{latc}{Latitude Circ.}
\printba{lest}{LEST Found., Annu. Rep.}
\printba{lmp}{Lett. Math. Phys.}
\printba{lob}{Lick Obs. Bull.}
\printba{mg}{Manuscr. Geod.}
\printba{ml}{Mater. Lett.}
\printba{mrb}{Mater. Res. Bull.}
\printba{mfs}{Mem. Fac. Sci., Kyoto Univ., Ser. Phys., Astrophys., Geophys., Chem.}
\printba{msai}{Mem. Soc. Astron. Ital.}
\printba{merc}{Mercury}
\printba{mess}{Messenger}
\printba{met}{Meteoritics}
\printba{metka}{Meteoritika}
\printba{metro}{Metrologica}
\printba{mpb}{Minor Planet Bull.}
\printba{mpc}{Minor Planet Circ.}
\printba{masbt}{Mitt. Archenold--Sternw. Berlin Treptow}
\printba{mag}{Mitt. Astron. Ges.}
\printba{mlotud}{Mitt. Lohrmann--Obs., Tech. Univ. Dresden}
\printba{mss}{Mitt. Sternw. Sonneberg}
\printba{mvs}{Mitt. Ver{\"a}nderliche Sterne}
\printba{mzpe}{Mitt. Zentralinst. Phys. Erde}
\printba{mpla}{Mod. Phys. Lett. A}
\printba{mn}{Mon. Not. R. Astron. Soc.}
\printba{mnassa}{Mon. Notes Astron. Soc. S. Afr.}
\printba{nint}{Nablyud. Iskusstv. Nebesn. Tel.}
\printba{nogb}{Nachr. Olbers--Ges. Bremen}
\printba{naoj}{Natl. Astron. Obs. (Jpn.), Repr.}
\printba{ng}{Natl. Geogr.}
\printba{nraoa}{Natl. Radio Astron. Obs., Repr., Ser. A}
\printba{nraob}{Natl. Radio Astron. Obs., Repr., Ser. B}
\printba{nat}{Nature}
\printba{natwis}{Naturwissenschaften}
\printba{ni}{Nauchn. Inf.}
\printba{nlasny}{News Lett. Astron. Soc. N.Y.}
\printba{nrodaour}{Nizamiah Rangapur Obs. Dep. Astron., Osmania Univ., Repr.}
\printba{nraow}{NRAO Workshop}
\printba{nimpra}{Nucl. Instrum. Methods Phys. Res., Sect. A}
\printba{nimprb}{Nucl. Instrum. Methods Phys. Res., Sect. B}
\printba{npa}{Nucl. Phys. A}
\printba{npb}{Nucl. Phys. B, Part. Phys.}
\printba{nca}{Nuovo Cimento A}
\printba{ncb}{Nuovo Cimento B}
\printba{ncc}{Nuovo Cimento C}
\printba{ot}{Obs. Trav.}
\printba{obs}{Observatory}
\printba{onl}{Occultation Newsl.}
\printba{ozvp}{{\"O}sterr. Z. Vermessungswes. Photogramm.}
\printba{oc}{Opt. Commun.}
\printba{oe}{Opt. Eng.}
\printba{olt}{Opt. Laser Technol.}
\printba{ol}{Opt. Lett.}
\printba{on}{Opt. News}
\printba{os}{Opt. Spectrosc.}
\printba{optik}{Optik}
\printba{ofeb}{Origins Life Evol. Biosphere}
\printba{orion}{Orion}
\printba{orione}{Orione}
\printba{oac}{Oss. Astrofis. Catania, Pubbl.}
\printba{pz}{Perem. Zvezdy}
\printba{pb}{Phys. Bl.}
\printba{pcm}{Phys. Chem. Miner.}
\printba{pfa}{Phys. Fluids, A}
\printba{pfb}{Phys. Fluids, B}
\printba{pla}{Phys. Lett. A}
\printba{plb}{Phys. Lett. B}
\printba{pr}{Phys. Rep.}
\printba{pra}{Phys. Rev. A}
\printba{prb}{Phys. Rev. B}
\printba{prc}{Phys. Rev. C}
\printba{prd}{Phys. Rev. D}
\printba{prl}{Phys. Rev. Lett.}
\printba{ps}{Phys. Scr.}
\printba{ptch}{Phys. Teach.}
\printba{ptod}{Phys. Today}
\printba{pha}{Physica A}
\printba{phb}{Physica B}
\printba{phd}{Physica D}
\printba{paz}{Pis'ma Astron. Zh.}
\printba{pss}{Planet. Space Sci.}
\printba{ppcf}{Plasma Phys. Controlled Fusion}
\printba{pa}{Postepy Astron.}
\printba{pram}{Pram\~{a}na}
\printba{pri}{Priroda}
\printba{pasa}{Proc. Astron. Soc. Aust.}
\printba{pi3e}{Proc. IEEE}
\printba{prsl}{Proc. R. Soc. London, Ser. A}
\printba{posp}{Proceedings of SPIE}
\printba{past}{Prog. Astron.}
\printba{ptp}{Prog. Theor. Phys.}
\printba{paicas}{Publ. Astron. Inst. Czech. Acad. Sci.}
\printba{pasj}{Publ. Astron. Soc. Jpn.}
\printba{pasp}{Publ. Astron. Soc. Pac.}
\printba{pbao}{Publ. Beijing Astron. Obs.}
\printba{pdaub}{Publ. Dep. Astron., Univ. Beogr.}
\printba{pdao}{Publ. Dom. Astrophys. Obs.}
\printba{pnlaoj}{Publ. Natl. Astron. Obs. Jpn.}
\printba{ppmo}{Publ. Purple Mt. Obs.}
\printba{psao}{Publ. Shaanxi Astron. Obs.}
\printba{pscds}{Publ. Sp\'ec. Cent. Donn\'ees Stellaires}
\printba{pyo}{Publ. Yunnan Obs.}
\printba{pag}{Pure Appl. Geophys.}
\printba{qbsa}{Q. Bull. Sol. Act.}
\printba{qjras}{Q. J. R. Astron. Soc.}
\printba{rad}{Radiant}
\printba{rsci}{Radio Sci.}
\printba{reche}{Recherche}
\printba{rpp}{Rep. Prog. Phys.}
\printba{ra}{Rev. Astron.}
\printba{rg}{Rev. Geophys.}
\printba{rmf}{Rev. Mex. Fis.}
\printba{rmp}{Rev. Mod. Phys.}
\printba{rrp}{Rev. Roum. Phys.}
\printba{rsi}{Rev. Sci. Instrum.}
\printba{rnc}{Riv. Nuovo Cimento}
\printba{rh}{\v{R}\'{\i}\v{s}e hv\v{e}zd}
\printba{saaoar}{S. Afr. Astron. Obs., Annu. Rep.}
\printba{saaoc}{S. Afr. Astron. Obs., Circ.}
\printba{saaon}{SAAO Newsl.}
\printba{sciam}{Sci. Am.}
\printba{scch}{Sci. China, Ser. A}
\printba{srtu}{Sci. Rep. T\^{o}hoku Univ., Eighth Ser.}
\printba{sc}{Science}
\printba{sar}{Sendai Astron. Rap.}
\printba{st}{Sidereal Times}
\printba{sky}{Sky Telesc.}
\printba{sbaavso}{Sol. Bull. (AAVSO)}
\printba{se}{Sol. Energy}
\printba{sp}{Sol. Phys.}
\printba{srd}{Sol. Radio Data}
\printba{ssres}{Sol. Syst. Res.}
\printba{sdb}{Soln. Dannye, Byull.}
\printba{sonne}{Sonne}
\printba{sbo}{Soobshch. Byurak. Obs.}
\printba{ssao}{Soobshch. Spets. Astrofiz. Obs.}
\printba{sost}{South. Stars}
\printba{sa}{Sov. Astron.}
\printba{sal}{Sov. Astron. Lett.}
\printba{sjot}{Sov. J. Opt. Technol.}
\printba{spd}{Sov. Phys. -- Dokl.}
\printba{spj}{Sov. Phys. -- JETP}
\printba{spu}{Sov. Phys. -- Usp.}
\printba{space}{Space}
\printba{ssr}{Space Sci. Rev.}
\printba{stsi}{Space Telesc. Sci. Inst., Newsl.}
\printba{sf}{Spaceflight}
\printba{sterne}{Sterne}
\printba{sw}{Sterne Weltraum}
\printba{sb}{Sternenbote}
\printba{strlast}{Strolling Astron.}
\printba{sgg}{Stud. Geophys. Geod.}
\printba{taot}{Tartu Astrof{\"u}{\"u}s. Obs. Teated}
\printba{tect}{Tectonophysics}
\printba{tellus}{Tellus, Ser. A}
\printba{tmf}{Teor. Mat. Fiz.}
\printba{tp}{Theor. Pap.}
\printba{tsb}{Time Serv. Bull.}
\printba{tmnr}{Tokyo Meteor Network Rep.}
\printba{tai}{Tsirk. Astron. Inst. (Tashkent)}
\printba{ufz}{Ukr. Fiz. Zh.}
\printba{ucla}{Universe Classroom}
\printba{uao}{Upps. Astron. Obs., Rep.}
\printba{urania}{Urania}
\printba{usnoc}{U.S. Nav. Obs., Circ.}
\printba{usnos4}{U.S. Nav. Obs., Ser. 4}
\printba{ufn}{Usp. Fiz. Nauk}
\printba{vsb}{Variable Star Bull.}
\printba{vasiona}{Vasiona}
\printba{varih}{Ver{\"o}ff. Astron. Rechen--Inst. Heidelb.}
\printba{vzpe}{Ver{\"o}ff. Zentralinst. Phys. Erde}
\printba{vesmir}{Vesm\'{\i}r}
\printba{vansssr}{Vestn. Akad. Nauk SSSR}
\printba{vku}{Vestn. Kiev. Univ., Astron.}
\printba{va}{Vistas Astron.}
\printba{wgn}{WGN}
\printba{wztud}{Wiss. Z. Tech. Univ. Dresden}
\printba{yama}{Yamamoto Circ.}
\printba{zamp}{Z. Angew. Math. Phys.}
\printba{zna}{Z. Naturforsch., A}
\printba{zpa}{Z. Phys., A}
\printba{zpc}{Z. Phys., C}
\printba{zv}{Zemlya Vselennaya}
\printba{zenit}{Zenit}
\printba{zetf}{Zh. Ehksp. Teor. Fiz.}

\end{document}
